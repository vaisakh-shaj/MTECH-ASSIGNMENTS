\documentclass{article}

\usepackage{fancyhdr}
\usepackage{extramarks}
\usepackage{amsmath}
\usepackage{amsthm}
\usepackage{amsfonts}
\usepackage{tikz}
\usepackage[plain]{algorithm}
\usepackage{algpseudocode}

\usetikzlibrary{automata,positioning}

%
% Basic Document Settings
%

\topmargin=-0.45in
\evensidemargin=0in
\oddsidemargin=0in
\textwidth=6.5in
\textheight=9.0in
\headsep=0.25in

\linespread{1.1}

\pagestyle{fancy}
\lhead{\hmwkAuthorName}
\chead{\hmwkClass\ (\hmwkClassInstructor\ \hmwkClassTime): \hmwkTitle}
\rhead{\firstxmark}
\lfoot{\lastxmark}
\cfoot{\thepage}

\renewcommand\headrulewidth{0.4pt}
\renewcommand\footrulewidth{0.4pt}

\setlength\parindent{0pt}

%
% Create Problem Sections
%

\newcommand{\enterProblemHeader}[1]{
    \nobreak\extramarks{}{Problem \arabic{#1} continued on next page\ldots}\nobreak{}
    \nobreak\extramarks{Problem \arabic{#1} (continued)}{Problem \arabic{#1} continued on next page\ldots}\nobreak{}
}

\newcommand{\exitProblemHeader}[1]{
    \nobreak\extramarks{Problem \arabic{#1} (continued)}{Problem \arabic{#1} continued on next page\ldots}\nobreak{}
    \stepcounter{#1}
    \nobreak\extramarks{Problem \arabic{#1}}{}\nobreak{}
}

\setcounter{secnumdepth}{0}
\newcounter{partCounter}
\newcounter{homeworkProblemCounter}
\setcounter{homeworkProblemCounter}{1}
\nobreak\extramarks{Problem \arabic{homeworkProblemCounter}}{}\nobreak{}

%
% Homework Problem Environment
%
% This environment takes an optional argument. When given, it will adjust the
% problem counter. This is useful for when the problems given for your
% assignment aren't sequential. See the last 3 problems of this template for an
% example.
%
\newenvironment{homeworkProblem}[1][-1]{
    \ifnum#1>0
        \setcounter{homeworkProblemCounter}{#1}
    \fi
    \section{Problem \arabic{homeworkProblemCounter}}
    \setcounter{partCounter}{1}
    \enterProblemHeader{homeworkProblemCounter}
}{
    \exitProblemHeader{homeworkProblemCounter}
}

%
% Homework Details
%   - Title
%   - Due date
%   - Class
%   - Section/Time
%   - Instructor
%   - Author
%

\newcommand{\hmwkTitle}{Assignment\ \#2}
\newcommand{\hmwkDueDate}{}
\newcommand{\hmwkClass}{Pattern Recognition And Machine Learning}
\newcommand{\hmwkClassTime}{}
\newcommand{\hmwkClassInstructor}{}
\newcommand{\hmwkAuthorName}{Vaisakh Shaj}

%
% Title Page
%

\title{
    \vspace{2in}
    \textmd{\textbf{\hmwkClass:\ \hmwkTitle}}\\
    %\normalsize\vspace{0.1in}\small{Due\ on\ \hmwkDueDate\ at 3:10pm}\\
    \vspace{0.1in}\large{\textit{\hmwkClassInstructor\ \hmwkClassTime}}
    \vspace{3in}
}

\author{\textbf{\hmwkAuthorName}}
\date{}

\renewcommand{\part}[1]{\textbf{\large Part \Alph{partCounter}}\stepcounter{partCounter}\\}

%
% Various Helper Commands
%

% Useful for algorithms
\newcommand{\alg}[1]{\textsc{\bfseries \footnotesize #1}}

% For derivatives
\newcommand{\deriv}[1]{\frac{\mathrm{d}}{\mathrm{d}x} (#1)}

% For partial derivatives
\newcommand{\pderiv}[2]{\frac{\partial}{\partial #1} (#2)}

% Integral dx
\newcommand{\dx}{\mathrm{d}x}

% Alias for the Solution section header
\newcommand{\solution}{\textbf{\large Solution}}

% Probability commands: Expectation, Variance, Covariance, Bias
\newcommand{\E}{\mathrm{E}}
\newcommand{\Var}{\mathrm{Var}}
\newcommand{\Cov}{\mathrm{Cov}}
\newcommand{\Bias}{\mathrm{Bias}}

\begin{document}

\maketitle

\pagebreak

\begin{homeworkProblem}
    \section{Show that the set X of all integers with metric defined by \textit{$d(m,n) = \mid  m-n \mid$}  is a complete metric space.}


    \textbf{Solution}\\

    Set of all integers X form a discrete space.Now the metric defined in it \textit{$d(m,n) = \mid m-n \mid$ }. Lets prove (X,d) form a valid metric space

    
1. d(x,y) $\geq$ 0 and finite 
\\ 2. d(x,y) = d(y,x)and d(x,y) = 0 iff x=y ,ie if x=y, d(x,y) = d(x,x) = $ \vert x-x \vert$ = 0
\\3.To prove triangle inequality, lets take x,y,z $\in$ X.
 
$d(x,z) =  \vert x-z \vert $ \\ $ = \vert x-y + y-z \vert$ 
$ \leq \vert x - y \vert + \vert y - z \vert$
\\$ = d(x,y) + d(y,z) $
\\$ \Rightarrow d(x,z) \leq d(x,y) + d(y,z) $  

So X is a metric space.\\

\textbf{ Proving Set of all Integers is Complete in the given metric space}\\


Unravelling the definition for a Cauchy sequence, we get that :

$ \forall \epsilon > 0 : \exists N : \forall m,n >N : d(a_n,a_m) < \epsilon $\\

and for $\epsilon= 1/2$ and with the given metric definition $d(m,n) = \mid m-n \mid$, we note that this must mean $d(a_n,a_m)=0$ (since otherwise it exceeds 1) i.e $a_n = a_m $ for some N and all $ m,n \geq N $.That is, $a_n$ is eventually constant.\\


Now there is an obvious guess for the limit of eventually constant sequences, and we conclude that set of integers with the Euclidean metric is complete.

\end{homeworkProblem}

\pagebreak

\begin{homeworkProblem}
    
\section{Show that \textit{$d(x,y)=\sqrt{|x-y|}$} defines a metric on the set of all real numbers.}

\textbf{Solution}

Lets prove (X,d) form a valid metric space

    
1. d(x,y) $\geq$ 0 and finite 
\\ 2. d(x,y) = d(y,x)and d(x,y) = 0 iff x=y ,ie if x=y, d(x,y) = d(x,x) = $ \sqrt{\vert x-x \vert}$ = 0
\\3.To prove triangle inequality, lets take x,y,z $\in$ X.
 
$d(x,z) =  \sqrt{\vert x-z \vert} $ \\ $ = \sqrt{\vert x-y + y-z \vert}$ 
\\$ \leq \sqrt{ \vert x - y \vert + \vert y - z \vert}$
\\$ \leq \sqrt{\vert x - y \vert} + \sqrt{\vert y - z \vert}$
\\$ = d(x,y) + d(y,z) $
\\$ \Rightarrow d(x,z) \leq d(x,y) + d(y,z) $  

So X is a metric space.\\

 
\end{homeworkProblem}
\pagebreak
\begin{homeworkProblem}

\section{Show that the closure \textit{$\bar{Y}$} of a subspace \textit{Y} of a normed space \textit{X} is again a vector space. }

\textbf{Solution}

It is sufficient to prove that $\alpha x + \beta y \in \bar{Y}$ where $\alpha $ and $\beta $ are in the underlying field $\mathbb{F}$ and x,y $\in \bar{Y.}$\\
Since x,y $\in \bar{Y} \exists     x_j,y_j \in X$ such that $x_j \rightarrow x$ and $y_j \rightarrow$ y. Since multiplication and addition are continuous $\alpha x_j + \beta y_j \rightarrow \alpha x + \beta y .$\\
\\
So, $\alpha x + \beta y \in \bar{Y}$.\\
Therefore $\bar{Y}$ is a subspace.
    
\end{homeworkProblem}

\pagebreak

\begin{homeworkProblem}

\section{Show that in an inner product space, $ x \perp y $ iff $ \parallel x + \alpha y \parallel \geq \parallel x \parallel \forall \alpha \in \mathbb{R}$} 
 
\textbf{Solution}\\

$\Vert x + \alpha y \Vert ^2 = \langle x+ \alpha y,x+ \alpha y\rangle $\\
$ = \langle x,x \rangle + \alpha \langle x,y \rangle + \alpha \langle y,x \rangle + \alpha^2 \langle y,y \rangle $\\
i.e , \\
$\Vert x + \alpha y \Vert ^2 = \langle x,x \rangle + 2\alpha \langle x,y \rangle + \alpha^2 \langle y,y \rangle $\\

$ = \|x\| ^2 + 2 \alpha \langle x,y \rangle +\alpha ^2 $---(1)\\

if x $\perp , \langle x,y \rangle = 0 $\\
 \\
$(1) \Rightarrow \Vert x + \alpha y \Vert ^2 = \Vert x \Vert^2 + \alpha^2 \vert y \Vert^2 $\\
$\Rightarrow \Vert x + \alpha y \Vert ^2 \geq \Vert x \Vert^2 $\\
$\Rightarrow \Vert x + \alpha y \Vert \geq \Vert x \Vert$\\

\textbf{hence proved the if part}\\

if $\Vert x + \alpha y \Vert ^2 \geq   \parallel x ^2 \parallel \forall \alpha \varepsilon \Re  $\\

$ \Rightarrow   2 \alpha \langle x,y \rangle +\alpha ^2 \geq 0 $ ,which is true \textbf{if} $ \langle  x,y \rangle = 0 $\\

\textbf{hence proved the only if part}
   
\end{homeworkProblem}

\pagebreak

\begin{homeworkProblem}
  
\section{Find  $\langle u,v \rangle$, where $v=(1+2i,3-i)^T$ , $u=(-2+i,4)^T$.}

\textbf{Solution}
$ langle u,v\rangle = \langle(-2+i,4),(1+2i,3-i)\rangle $ \\
for complex numbers $\langle (x1,x2)(y1,y2) \rangle = x1*\overline{y1}+x2*\overline{y2}$\\
$ = (-2+i)(1-2i) + 4(3+i)$\\
$ = -2+4i+i+2+12+4i$\\
$ = 9i+12$\ 
  
\end{homeworkProblem}
\pagebreak

\begin{homeworkProblem}

\section{Which of the following subsets of $\mathbb{R}^3$ constitute a subspace of $\mathbb{R}^3$ ? [x=$(\eta_1,\eta_2,\eta_3)^T$]}
\section{(a) All x with $\eta_1 = \eta_2$ and $\eta_3 = 0$.}
\section{(b) All x with $\eta_1 = \eta_2+1$}



\textbf{Solution}

a)\\
Let Z=\{All x with $\eta_1 = \eta_2$ and $\eta_3 = 0$\}.\\
Consider X=$(x,x,0)$,Y=$(y,y,0)$ $\in Z$\\
$X+Y = (x+y,x+y,0) \in Z $\\
$ \alpha X = (\alpha x,\alpha x,0) \in Z $\\
Thus Z is closed under addition and scalar multiplication,hence it is a subspace of $\mathbb{R}^3$.

b)\\
Let Z=\{All x with $\eta_1 = \eta_2+1$\}.\\
Consider X=$(x+1,x,p)$,Y=$(y+1,y,q)$ $\in Z$ where p,q$\in \mathbb{R}$.\\
$X+Y = (x+y+2,x+y,p+q)) \notin Z  $\\
because $\eta_1 \neq \eta_2+1$ is violated here. Hence Z is not closed under addition. So it is not a subspace of $\mathbb{R}^3$.

\end{homeworkProblem}

\pagebreak

\begin{homeworkProblem}
\section{Show that $\|x\|$ is the distance from x to 0.}

\textbf{Solution}

Considering $\|x||_2$ ,
d(x,y)= $\|x-y\|$ is clearly a metric space.\\

Let y=0.\\
Therefore d(x,0)= $\|x-y\|$ = distance between x and origin = $\|x\|$.     
\end{homeworkProblem}

\pagebreak

\begin{homeworkProblem}
 \section{If an inner product space $\langle x,u \rangle$ = $\langle x,v \rangle$ for all x, show that u=v.}
 
 \textbf{Solution}  
 
 Given $\langle x,u \rangle$ = $\langle x,v \rangle$\\
i.e, $\langle x,u \rangle$ - $\langle x,v \rangle = 0$\\
$\Rightarrow \langle x,u-v \rangle = 0$-----(1)\\

we have to prove its true $\forall x$

take x= u-v

(1)$ \Rightarrow \langle u-v,u-v \rangle = 0$\\
$\Rightarrow \parallel u-v \parallel ^2 = 0 $\\
$ \Rightarrow \parallel u-v \parallel = 0 $\\

$ \Rightarrow u=v $

Hence Proved

\end{homeworkProblem}

\pagebreak

\begin{homeworkProblem}
\section{Prove that $\Vert T_1T_2 \Vert \leq \Vert T_1 \Vert \Vert T_2 \Vert $ ;$ \Vert T^n \Vert \leq \Vert T \Vert^n $}

\textbf{Solutions}\\

\textbf{To Prove $\Vert T_1T_2 \Vert \leq \Vert T_1 \Vert \Vert T_2 \Vert $}


A vector norm and its induced matrix norm satisfy   the inequality\\
$\Vert Tx \Vert \leq \Vert T \Vert \Vert x \Vert$ \\
Replace x by $T_2$x,\\
$\Vert T_1T_2x \Vert \leq \Vert T_1 \Vert \Vert T_2 \Vert \Vert x \Vert$\\
For nonzero x we can divide both sides by the positive number $\Vert x \Vert$ and can conclude that\\
$ \Vert T_1T_2\Vert  = max_x \frac{\Vert T_1T_2x\Vert}{\Vert x\Vert} \leq \Vert T_1 \Vert \Vert T_2 \Vert $
\\
Hence Proved
\\

\textbf{To prove $ \Vert T^n \Vert \leq \Vert T \Vert^n $ }\\

if $T_1 = T_2 $\\
$ \Vert TT \Vert = \Vert T^2 \Vert \leq \Vert T \Vert^2 $\\
similiarly,
$ \Vert T.......T \Vert  \leq \Vert T \Vert \Vert T........T \Vert $\\
$ \Vert T \Vert^n  \leq \Vert T \Vert \Vert T \Vert \Vert T........T \Vert $\\
Therefore $ \Vert T^n \Vert \leq \Vert T \Vert^n $
Hence, PROVED.
  
\end{homeworkProblem} 

\pagebreak 
 
\begin{homeworkProblem}

\section{For a real inner product space prove that $\langle x,y \rangle$ = $\frac{1}{4}(||x+y||^2-||x-y||^2 )$} 

\textbf{Solution}

$\Vert x+y \Vert^2$\\
 = $\langle x+y, x+y \rangle$\\
$ = \langle x,x \rangle + \langle x,y \rangle + \langle y,x \rangle+ \langle y,y \rangle $\\
$ = \langle x,x \rangle + 2\langle x,y \rangle + \langle y,y \rangle $\\
$ = \Vert x \Vert^2 + 2\langle x,y \rangle + \Vert y \Vert^2 $ \textbf{-----(1)} \\
similarly,\\
$\Vert x-y \Vert^2$ = $\Vert x \Vert^2 - 2\langle x,y \rangle + \Vert y \Vert^2 $ \textbf{----(2)}\\

\textbf{(1)-(2)} $\Rightarrow \Vert x+y \Vert^2 - \Vert x-y \Vert^2 = 4\langle x,y \rangle $ \\

$\Rightarrow \langle x,y \rangle = 1/4 (\Vert x+y \Vert^2 - \Vert x-y \Vert^2)$

\end{homeworkProblem}

\pagebreak

\begin{homeworkProblem}
  \section{11. Define T$\colon \mathbb{R}^2 \longrightarrow \mathbb{R}^2$ by T(x,y)=(x,0). Is T a linear operator ?} 
  
An linear operator satisties the following conditions,
$i. f(x+y) = f(x) + f(y), \forall \space  x,y\space \epsilon \space Z, Z\space is\space the\space vector\space space$\\
$ii. f(\alpha x) = \alpha f(x) , \forall \space \alpha \space \epsilon \space F(field), \space x \space \epsilon \space Z$\\
Checking the conditions on the operator defined,\\
$f(x,y) = (x,0) $\\
$i. f(x_1,y_1) + f(x_2,y_2) = (x_1 , 0) + (x_2, 0)$
$= (x_1 + x_2, 0) \space = \space f(x_1 + x_2, y_1 + y_2)$\\
$ii. f(\alpha x, y) = (\alpha x, 0) = \alpha (x, 0) = \alpha f(x,0) = \alpha f(x,y)$\\

Thus T a linear operator . 
\end{homeworkProblem}

\pagebreak

\begin{homeworkProblem}
\section{Show that a discrete metric space is complete.}  
 
\textbf{Solution}\\
Let (X,d) be the discrete metric space. The standard discrete metric d is defined as d(x,y) = 0 if x=y 
		   						d(x,y) = 1 otherwise

Unravelling the definition for a Cauchy sequence, we get that :

$ \forall \epsilon > 0 : \exists N : \forall m,n >N : d(a_n,a_m) < \epsilon $\\

and for $\epsilon= 1/2$ and with the given metric definition \textbf{d(m,n) = 0 if m=n and d(m,n) = 1 otherwise}, we note that this must mean $d(a_n,a_m)=0$ (since otherwise it exceeds 1) i.e $a_n = a_m $ for some N and all $ m,n \geq N $.That is, $a_n$ is eventually constant.\\


Now there is an obvious guess for the limit of eventually constant sequences, and we conclude that a discrete metric space is always complete.

\end{homeworkProblem}

\pagebreak

\begin{homeworkProblem}
  
\section{Describe Weistrass approximation theorem.}

\textbf{Solution}
  
The set W of all polynomials with real coefficients is dense in the real space C[a,b].\\
Hence for every x $\in$ [a,b] and given $ \in > 0  $ $\exists $ a polynomial p such that $\vert x(t) - p(t)\vert < \in $ for all t $\in$ [a,b]\\


\end{homeworkProblem}

\end{document}